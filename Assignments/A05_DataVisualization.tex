\documentclass[]{article}
\usepackage{lmodern}
\usepackage{amssymb,amsmath}
\usepackage{ifxetex,ifluatex}
\usepackage{fixltx2e} % provides \textsubscript
\ifnum 0\ifxetex 1\fi\ifluatex 1\fi=0 % if pdftex
  \usepackage[T1]{fontenc}
  \usepackage[utf8]{inputenc}
\else % if luatex or xelatex
  \ifxetex
    \usepackage{mathspec}
  \else
    \usepackage{fontspec}
  \fi
  \defaultfontfeatures{Ligatures=TeX,Scale=MatchLowercase}
\fi
% use upquote if available, for straight quotes in verbatim environments
\IfFileExists{upquote.sty}{\usepackage{upquote}}{}
% use microtype if available
\IfFileExists{microtype.sty}{%
\usepackage{microtype}
\UseMicrotypeSet[protrusion]{basicmath} % disable protrusion for tt fonts
}{}
\usepackage[margin=2.54cm]{geometry}
\usepackage{hyperref}
\hypersetup{unicode=true,
            pdftitle={Assignment 5: Data Visualization},
            pdfauthor={Ying Wei Jong},
            pdfborder={0 0 0},
            breaklinks=true}
\urlstyle{same}  % don't use monospace font for urls
\usepackage{color}
\usepackage{fancyvrb}
\newcommand{\VerbBar}{|}
\newcommand{\VERB}{\Verb[commandchars=\\\{\}]}
\DefineVerbatimEnvironment{Highlighting}{Verbatim}{commandchars=\\\{\}}
% Add ',fontsize=\small' for more characters per line
\usepackage{framed}
\definecolor{shadecolor}{RGB}{248,248,248}
\newenvironment{Shaded}{\begin{snugshade}}{\end{snugshade}}
\newcommand{\KeywordTok}[1]{\textcolor[rgb]{0.13,0.29,0.53}{\textbf{#1}}}
\newcommand{\DataTypeTok}[1]{\textcolor[rgb]{0.13,0.29,0.53}{#1}}
\newcommand{\DecValTok}[1]{\textcolor[rgb]{0.00,0.00,0.81}{#1}}
\newcommand{\BaseNTok}[1]{\textcolor[rgb]{0.00,0.00,0.81}{#1}}
\newcommand{\FloatTok}[1]{\textcolor[rgb]{0.00,0.00,0.81}{#1}}
\newcommand{\ConstantTok}[1]{\textcolor[rgb]{0.00,0.00,0.00}{#1}}
\newcommand{\CharTok}[1]{\textcolor[rgb]{0.31,0.60,0.02}{#1}}
\newcommand{\SpecialCharTok}[1]{\textcolor[rgb]{0.00,0.00,0.00}{#1}}
\newcommand{\StringTok}[1]{\textcolor[rgb]{0.31,0.60,0.02}{#1}}
\newcommand{\VerbatimStringTok}[1]{\textcolor[rgb]{0.31,0.60,0.02}{#1}}
\newcommand{\SpecialStringTok}[1]{\textcolor[rgb]{0.31,0.60,0.02}{#1}}
\newcommand{\ImportTok}[1]{#1}
\newcommand{\CommentTok}[1]{\textcolor[rgb]{0.56,0.35,0.01}{\textit{#1}}}
\newcommand{\DocumentationTok}[1]{\textcolor[rgb]{0.56,0.35,0.01}{\textbf{\textit{#1}}}}
\newcommand{\AnnotationTok}[1]{\textcolor[rgb]{0.56,0.35,0.01}{\textbf{\textit{#1}}}}
\newcommand{\CommentVarTok}[1]{\textcolor[rgb]{0.56,0.35,0.01}{\textbf{\textit{#1}}}}
\newcommand{\OtherTok}[1]{\textcolor[rgb]{0.56,0.35,0.01}{#1}}
\newcommand{\FunctionTok}[1]{\textcolor[rgb]{0.00,0.00,0.00}{#1}}
\newcommand{\VariableTok}[1]{\textcolor[rgb]{0.00,0.00,0.00}{#1}}
\newcommand{\ControlFlowTok}[1]{\textcolor[rgb]{0.13,0.29,0.53}{\textbf{#1}}}
\newcommand{\OperatorTok}[1]{\textcolor[rgb]{0.81,0.36,0.00}{\textbf{#1}}}
\newcommand{\BuiltInTok}[1]{#1}
\newcommand{\ExtensionTok}[1]{#1}
\newcommand{\PreprocessorTok}[1]{\textcolor[rgb]{0.56,0.35,0.01}{\textit{#1}}}
\newcommand{\AttributeTok}[1]{\textcolor[rgb]{0.77,0.63,0.00}{#1}}
\newcommand{\RegionMarkerTok}[1]{#1}
\newcommand{\InformationTok}[1]{\textcolor[rgb]{0.56,0.35,0.01}{\textbf{\textit{#1}}}}
\newcommand{\WarningTok}[1]{\textcolor[rgb]{0.56,0.35,0.01}{\textbf{\textit{#1}}}}
\newcommand{\AlertTok}[1]{\textcolor[rgb]{0.94,0.16,0.16}{#1}}
\newcommand{\ErrorTok}[1]{\textcolor[rgb]{0.64,0.00,0.00}{\textbf{#1}}}
\newcommand{\NormalTok}[1]{#1}
\usepackage{graphicx,grffile}
\makeatletter
\def\maxwidth{\ifdim\Gin@nat@width>\linewidth\linewidth\else\Gin@nat@width\fi}
\def\maxheight{\ifdim\Gin@nat@height>\textheight\textheight\else\Gin@nat@height\fi}
\makeatother
% Scale images if necessary, so that they will not overflow the page
% margins by default, and it is still possible to overwrite the defaults
% using explicit options in \includegraphics[width, height, ...]{}
\setkeys{Gin}{width=\maxwidth,height=\maxheight,keepaspectratio}
\IfFileExists{parskip.sty}{%
\usepackage{parskip}
}{% else
\setlength{\parindent}{0pt}
\setlength{\parskip}{6pt plus 2pt minus 1pt}
}
\setlength{\emergencystretch}{3em}  % prevent overfull lines
\providecommand{\tightlist}{%
  \setlength{\itemsep}{0pt}\setlength{\parskip}{0pt}}
\setcounter{secnumdepth}{0}
% Redefines (sub)paragraphs to behave more like sections
\ifx\paragraph\undefined\else
\let\oldparagraph\paragraph
\renewcommand{\paragraph}[1]{\oldparagraph{#1}\mbox{}}
\fi
\ifx\subparagraph\undefined\else
\let\oldsubparagraph\subparagraph
\renewcommand{\subparagraph}[1]{\oldsubparagraph{#1}\mbox{}}
\fi

%%% Use protect on footnotes to avoid problems with footnotes in titles
\let\rmarkdownfootnote\footnote%
\def\footnote{\protect\rmarkdownfootnote}

%%% Change title format to be more compact
\usepackage{titling}

% Create subtitle command for use in maketitle
\newcommand{\subtitle}[1]{
  \posttitle{
    \begin{center}\large#1\end{center}
    }
}

\setlength{\droptitle}{-2em}

  \title{Assignment 5: Data Visualization}
    \pretitle{\vspace{\droptitle}\centering\huge}
  \posttitle{\par}
    \author{Ying Wei Jong}
    \preauthor{\centering\large\emph}
  \postauthor{\par}
    \date{}
    \predate{}\postdate{}
  

\begin{document}
\maketitle

\subsection{OVERVIEW}\label{overview}

This exercise accompanies the lessons in Environmental Data Analytics
(ENV872L) on data wrangling.

\subsection{Directions}\label{directions}

\begin{enumerate}
\def\labelenumi{\arabic{enumi}.}
\tightlist
\item
  Change ``Student Name'' on line 3 (above) with your name.
\item
  Use the lesson as a guide. It contains code that can be modified to
  complete the assignment.
\item
  Work through the steps, \textbf{creating code and output} that fulfill
  each instruction.
\item
  Be sure to \textbf{answer the questions} in this assignment document.
  Space for your answers is provided in this document and is indicated
  by the ``\textgreater{}'' character. If you need a second paragraph be
  sure to start the first line with ``\textgreater{}''. You should
  notice that the answer is highlighted in green by RStudio.
\item
  When you have completed the assignment, \textbf{Knit} the text and
  code into a single PDF file. You will need to have the correct
  software installed to do this (see Software Installation Guide) Press
  the \texttt{Knit} button in the RStudio scripting panel. This will
  save the PDF output in your Assignments folder.
\item
  After Knitting, please submit the completed exercise (PDF file) to the
  dropbox in Sakai. Please add your last name into the file name (e.g.,
  ``Salk\_A04\_DataWrangling.pdf'') prior to submission.
\end{enumerate}

The completed exercise is due on Tuesday, 19 February, 2019 before class
begins.

\subsection{Set up your session}\label{set-up-your-session}

\begin{enumerate}
\def\labelenumi{\arabic{enumi}.}
\item
  Set up your session. Upload the NTL-LTER processed data files for
  chemistry/physics for Peter and Paul Lakes (tidy and gathered), the
  USGS stream gauge dataset, and the EPA Ecotox dataset for
  Neonicotinoids.
\item
  Make sure R is reading dates as date format, not something else (hint:
  remember that dates were an issue for the USGS gauge data).
\end{enumerate}

\begin{Shaded}
\begin{Highlighting}[]
\CommentTok{#1}
\NormalTok{PeterPaulTidy <-}\StringTok{ }\KeywordTok{read.csv}\NormalTok{(}\StringTok{"./Data/Processed/NTL-LTER_Lake_Chemistry_Nutrients_PeterPaul_Processed.csv"}\NormalTok{, }\DataTypeTok{header=}\NormalTok{T)}
\NormalTok{PeterPaulGathered <-}\StringTok{ }\KeywordTok{read.csv}\NormalTok{(}\StringTok{"./Data/Processed/NTL-LTER_Lake_Nutrients_PeterPaulGathered_Processed.csv"}\NormalTok{, }\DataTypeTok{header=}\NormalTok{T)}
\NormalTok{USGS.data <-}\StringTok{ }\KeywordTok{read.csv}\NormalTok{(}\StringTok{"./Data/Raw/USGS_Site02085000_Flow_Raw.csv"}\NormalTok{, }\DataTypeTok{header=}\NormalTok{T)}
\NormalTok{Ecotox <-}\StringTok{ }\KeywordTok{read.csv}\NormalTok{(}\StringTok{"./Data/Raw/ECOTOX_Neonicotinoids_Mortality_raw.csv"}\NormalTok{)}

\CommentTok{#2}
\KeywordTok{str}\NormalTok{(PeterPaulTidy) }
\end{Highlighting}
\end{Shaded}

\begin{verbatim}
## 'data.frame':    23372 obs. of  14 variables:
##  $ lakename       : Factor w/ 2 levels "Paul Lake","Peter Lake": 1 1 1 1 1 1 1 1 1 1 ...
##  $ daynum         : int  148 148 148 148 148 148 148 148 148 148 ...
##  $ year4          : int  1984 1984 1984 1984 1984 1984 1984 1984 1984 1984 ...
##  $ sampledate     : Factor w/ 1105 levels "1984-05-27","1984-05-28",..: 1 1 1 1 1 1 1 1 1 1 ...
##  $ depth          : num  0 0.25 0.5 0.75 1 1.5 2 3 4 5 ...
##  $ temperature_C  : num  14.5 NA NA NA 14.5 NA 14.2 11 7 6.1 ...
##  $ dissolvedOxygen: num  9.5 NA NA NA 8.8 NA 8.6 11.5 11.9 2.5 ...
##  $ irradianceWater: num  1750 1550 1150 975 870 610 420 220 100 34 ...
##  $ irradianceDeck : num  1620 1620 1620 1620 1620 1620 1620 1620 1620 1620 ...
##  $ tn_ug          : num  NA NA NA NA NA NA NA NA NA NA ...
##  $ tp_ug          : num  NA NA NA NA NA NA NA NA NA NA ...
##  $ nh34           : num  NA NA NA NA NA NA NA NA NA NA ...
##  $ no23           : num  NA NA NA NA NA NA NA NA NA NA ...
##  $ po4            : num  NA NA NA NA NA NA NA NA NA NA ...
\end{verbatim}

\begin{Shaded}
\begin{Highlighting}[]
\KeywordTok{str}\NormalTok{(PeterPaulGathered)}
\end{Highlighting}
\end{Shaded}

\begin{verbatim}
## 'data.frame':    7997 obs. of  7 variables:
##  $ lakename     : Factor w/ 2 levels "Paul Lake","Peter Lake": 1 1 1 1 1 1 2 2 2 2 ...
##  $ daynum       : int  140 140 140 140 140 140 140 140 140 140 ...
##  $ year4        : int  1991 1991 1991 1991 1991 1991 1991 1991 1991 1991 ...
##  $ sampledate   : Factor w/ 778 levels "1991-05-20","1991-05-27",..: 1 1 1 1 1 1 1 1 1 1 ...
##  $ depth        : num  0 0.85 1.75 3 4 6 0 1 2.25 3.5 ...
##  $ nutrient     : Factor w/ 5 levels "nh34","no23",..: 4 4 4 4 4 4 4 4 4 4 ...
##  $ concentration: num  538 285 399 453 363 583 352 356 364 582 ...
\end{verbatim}

\begin{Shaded}
\begin{Highlighting}[]
\KeywordTok{str}\NormalTok{(USGS.data) }
\end{Highlighting}
\end{Shaded}

\begin{verbatim}
## 'data.frame':    33216 obs. of  15 variables:
##  $ agency_cd             : Factor w/ 1 level "USGS": 1 1 1 1 1 1 1 1 1 1 ...
##  $ site_no               : int  2085000 2085000 2085000 2085000 2085000 2085000 2085000 2085000 2085000 2085000 ...
##  $ datetime              : Factor w/ 33216 levels "1/1/00","1/1/01",..: 20 1021 2022 2295 2386 2477 2568 2659 2750 111 ...
##  $ X165986_00060_00001   : num  74 61 56 54 48 47 44 41 44 57 ...
##  $ X165986_00060_00001_cd: Factor w/ 4 levels "","A","A:e","P": 2 2 2 2 2 2 2 2 2 2 ...
##  $ X165987_00060_00002   : num  NA NA NA NA NA NA NA NA NA NA ...
##  $ X165987_00060_00002_cd: Factor w/ 3 levels "","A","P": 1 1 1 1 1 1 1 1 1 1 ...
##  $ X84936_00060_00003    : num  NA NA NA NA NA NA NA NA NA NA ...
##  $ X84936_00060_00003_cd : Factor w/ 3 levels "","A","P": 1 1 1 1 1 1 1 1 1 1 ...
##  $ X84937_00065_00001    : num  NA NA NA NA NA NA NA NA NA NA ...
##  $ X84937_00065_00001_cd : Factor w/ 3 levels "","A","P": 1 1 1 1 1 1 1 1 1 1 ...
##  $ X84938_00065_00002    : num  NA NA NA NA NA NA NA NA NA NA ...
##  $ X84938_00065_00002_cd : Factor w/ 3 levels "","A","P": 1 1 1 1 1 1 1 1 1 1 ...
##  $ X84939_00065_00003    : num  NA NA NA NA NA NA NA NA NA NA ...
##  $ X84939_00065_00003_cd : Factor w/ 3 levels "","A","P": 1 1 1 1 1 1 1 1 1 1 ...
\end{verbatim}

\begin{Shaded}
\begin{Highlighting}[]
\KeywordTok{str}\NormalTok{(Ecotox)}
\end{Highlighting}
\end{Shaded}

\begin{verbatim}
## 'data.frame':    1283 obs. of  13 variables:
##  $ CAS.No.          : int  138261413 111988499 138261413 138261413 111988499 111988499 111988499 111988499 138261413 138261413 ...
##  $ Chemical.Name    : Factor w/ 9 levels "Acetamiprid",..: 4 8 4 4 8 8 8 8 4 4 ...
##  $ Species.Name     : Factor w/ 172 levels "Acipenser transmontanus",..: 54 86 54 43 54 54 54 54 43 98 ...
##  $ Common.Name      : Factor w/ 124 levels "Alderfly","Alfalfa Plant Bug",..: 68 97 68 68 68 68 68 68 68 97 ...
##  $ Effect           : Factor w/ 1 level "Mortality": 1 1 1 1 1 1 1 1 1 1 ...
##  $ Measurement      : Factor w/ 1 level "Mortality": 1 1 1 1 1 1 1 1 1 1 ...
##  $ Endpoint         : Factor w/ 23 levels "EC10","EC50",..: 5 23 9 5 5 5 5 9 9 20 ...
##  $ Dur..Std.        : num  28 7 28 28 21 28 14 28 28 4 ...
##  $ Conc..Type       : Factor w/ 3 levels "Active ingredient",..: 2 1 2 2 1 1 1 1 2 1 ...
##  $ Conc..Mean..Std. : num  0.000041 0.00007 0.000195 0.000235 0.00024 0.00027 0.0003 0.0003 0.000316 0.00035 ...
##  $ Conc..Units..Std.: Factor w/ 16 levels "AI mg/kg bdwt",..: 4 4 4 4 4 4 4 4 4 4 ...
##  $ Pub..Year        : int  2013 2017 2013 2013 2016 2016 2016 2016 2013 1992 ...
##  $ Citation         : Factor w/ 198 levels "Aaen,S.M., L.A. Hamre, and T.E. Horsberg. A Screening of Medicinal Compounds for Their Effect on Egg Strings an"| __truncated__,..: 137 57 137 137 173 173 173 173 137 170 ...
\end{verbatim}

\begin{Shaded}
\begin{Highlighting}[]
\CommentTok{#Need to change date for PeterPaulTidy, PeterPaulGathered and USGS.data}
\CommentTok{#It is funny that I did not realize until now, but in the as.Date function, I need to make sure that the format i specified is the same format with what the raw.data has, otherwise R cannot understand it.}
\NormalTok{PeterPaulTidy}\OperatorTok{$}\NormalTok{sampledate <-}\StringTok{ }\KeywordTok{as.Date}\NormalTok{(PeterPaulTidy}\OperatorTok{$}\NormalTok{sampledate, }\DataTypeTok{format =} \StringTok{"%m/%d/%y"}\NormalTok{)}
\NormalTok{PeterPaulGathered}\OperatorTok{$}\NormalTok{sampledate <-}\StringTok{ }\KeywordTok{as.Date}\NormalTok{(PeterPaulGathered}\OperatorTok{$}\NormalTok{sampledate, }\DataTypeTok{format =} \StringTok{"%Y-%m-%d"}\NormalTok{)}
\NormalTok{USGS.data}\OperatorTok{$}\NormalTok{datetime <-}\StringTok{ }\KeywordTok{as.Date}\NormalTok{(USGS.data}\OperatorTok{$}\NormalTok{datetime, }\DataTypeTok{format =} \StringTok{"%m/%d/%y"}\NormalTok{)}
\NormalTok{USGS.data}\OperatorTok{$}\NormalTok{datetime <-}\StringTok{ }\KeywordTok{format}\NormalTok{(USGS.data}\OperatorTok{$}\NormalTok{datetime, }\DataTypeTok{format =} \StringTok{"%y%m%d"}\NormalTok{)}
\NormalTok{create.early.dates <-}\StringTok{ }\NormalTok{(}\ControlFlowTok{function}\NormalTok{(d) \{}
       \KeywordTok{paste0}\NormalTok{(}\KeywordTok{ifelse}\NormalTok{(d }\OperatorTok{>}\StringTok{ }\DecValTok{181231}\NormalTok{,}\StringTok{"19"}\NormalTok{,}\StringTok{"20"}\NormalTok{),d)}
\NormalTok{       \})}
\NormalTok{USGS.data}\OperatorTok{$}\NormalTok{datetime <-}\StringTok{ }\KeywordTok{create.early.dates}\NormalTok{(USGS.data}\OperatorTok{$}\NormalTok{datetime)}
\NormalTok{USGS.data}\OperatorTok{$}\NormalTok{datetime <-}\StringTok{ }\KeywordTok{as.Date}\NormalTok{(USGS.data}\OperatorTok{$}\NormalTok{datetime, }\DataTypeTok{format =} \StringTok{"%Y%m%d"}\NormalTok{)}
\end{Highlighting}
\end{Shaded}

\subsection{Define your theme}\label{define-your-theme}

\begin{enumerate}
\def\labelenumi{\arabic{enumi}.}
\setcounter{enumi}{2}
\tightlist
\item
  Build a theme and set it as your default theme.
\end{enumerate}

\begin{Shaded}
\begin{Highlighting}[]
\CommentTok{#3}
\KeywordTok{library}\NormalTok{(ggplot2)}
\NormalTok{my.theme <-}\StringTok{ }\KeywordTok{theme_bw}\NormalTok{(}\DataTypeTok{base_size =} \DecValTok{12}\NormalTok{) }\OperatorTok{+}\StringTok{ }
\StringTok{  }\KeywordTok{theme}\NormalTok{(}\DataTypeTok{axis.text=}\KeywordTok{element_text}\NormalTok{(}\DataTypeTok{color=}\StringTok{"gray0"}\NormalTok{), }\DataTypeTok{legend.position =} \StringTok{"right"}\NormalTok{)}
\KeywordTok{theme_set}\NormalTok{(my.theme)}
\end{Highlighting}
\end{Shaded}

\subsection{Create graphs}\label{create-graphs}

For numbers 4-7, create graphs that follow best practices for data
visualization. To make your graphs ``pretty,'' ensure your theme, color
palettes, axes, and legends are edited to your liking.

Hint: a good way to build graphs is to make them ugly first and then
create more code to make them pretty.

\begin{enumerate}
\def\labelenumi{\arabic{enumi}.}
\setcounter{enumi}{3}
\tightlist
\item
  {[}NTL-LTER{]} Plot total phosphorus by phosphate, with separate
  aesthetics for Peter and Paul lakes. Add a line of best fit and color
  it black.
\end{enumerate}

\begin{Shaded}
\begin{Highlighting}[]
\CommentTok{#4}
\KeywordTok{ggplot}\NormalTok{(PeterPaulTidy, }\KeywordTok{aes}\NormalTok{(}\DataTypeTok{y=}\NormalTok{tp_ug, }\DataTypeTok{x=}\NormalTok{po4,}\DataTypeTok{color =}\NormalTok{ lakename,}\DataTypeTok{shape=}\NormalTok{lakename)) }\OperatorTok{+}
\StringTok{  }\KeywordTok{geom_point}\NormalTok{() }\OperatorTok{+}
\StringTok{  }\KeywordTok{xlab}\NormalTok{(}\StringTok{"Phosphate"}\NormalTok{) }\OperatorTok{+}
\StringTok{  }\KeywordTok{ylab}\NormalTok{(}\StringTok{"Total Phosphorus"}\NormalTok{) }\OperatorTok{+}\StringTok{ }
\StringTok{  }\KeywordTok{scale_color_manual}\NormalTok{(}\DataTypeTok{values =} \KeywordTok{c}\NormalTok{(}\StringTok{"rosybrown1"}\NormalTok{, }\StringTok{"aquamarine"}\NormalTok{)) }\OperatorTok{+}
\StringTok{  }\KeywordTok{geom_smooth}\NormalTok{(}\DataTypeTok{method =}\NormalTok{ lm, }\DataTypeTok{formula =}\NormalTok{ y}\OperatorTok{~}\NormalTok{x,}\DataTypeTok{col=}\StringTok{"black"}\NormalTok{)}
\end{Highlighting}
\end{Shaded}

\begin{verbatim}
## Warning: Removed 22309 rows containing non-finite values (stat_smooth).
\end{verbatim}

\begin{verbatim}
## Warning: Removed 22309 rows containing missing values (geom_point).
\end{verbatim}

\includegraphics{A05_DataVisualization_files/figure-latex/unnamed-chunk-3-1.pdf}

\begin{Shaded}
\begin{Highlighting}[]
\CommentTok{#Shall I take out that outlier? Why are there two lines?}
\end{Highlighting}
\end{Shaded}

\begin{enumerate}
\def\labelenumi{\arabic{enumi}.}
\setcounter{enumi}{4}
\tightlist
\item
  {[}NTL-LTER{]} Plot nutrients by date for Peter Lake, with separate
  colors for each depth. Facet your graph by the nutrient type.
\end{enumerate}

\begin{Shaded}
\begin{Highlighting}[]
\CommentTok{#5}
\KeywordTok{library}\NormalTok{(RColorBrewer)}

\NormalTok{nutrient_names <-}\StringTok{ }\KeywordTok{list}\NormalTok{(}
  \StringTok{"nh34"}\NormalTok{=}\StringTok{"Ammonium"}\NormalTok{,}
  \StringTok{"no23"}\NormalTok{=}\StringTok{"Nitrate"}\NormalTok{,}
  \StringTok{"po4"}\NormalTok{=}\StringTok{"Phosphate"}\NormalTok{,}
  \StringTok{"tn_ug"}\NormalTok{=}\StringTok{"Total Nitrogen"}\NormalTok{,}
  \StringTok{"tp_ug"}\NormalTok{=}\StringTok{"Total Phosphorus"} 
\NormalTok{)}

\NormalTok{nutrient_labeller <-}\StringTok{ }\ControlFlowTok{function}\NormalTok{(variable,value)\{}
  \KeywordTok{return}\NormalTok{(nutrient_names[value])}
\NormalTok{\}}
\KeywordTok{ggplot}\NormalTok{(PeterPaulGathered, }\KeywordTok{aes}\NormalTok{(}\DataTypeTok{y=}\NormalTok{concentration, }\DataTypeTok{x=}\NormalTok{sampledate, }\DataTypeTok{color=}\NormalTok{depth))}\OperatorTok{+}
\StringTok{  }\KeywordTok{geom_point}\NormalTok{()}\OperatorTok{+}
\StringTok{  }\KeywordTok{facet_grid}\NormalTok{(PeterPaulGathered}\OperatorTok{$}\NormalTok{nutrient, }\DataTypeTok{labeller =}\NormalTok{ nutrient_labeller)}\OperatorTok{+}
\StringTok{  }\CommentTok{#facet_wrap(vars(nutrient), nrow=5) +}
\StringTok{  }\KeywordTok{xlab}\NormalTok{(}\StringTok{"Dates"}\NormalTok{)}\OperatorTok{+}
\StringTok{  }\KeywordTok{ylab}\NormalTok{(}\StringTok{"Nutrient concentration"}\NormalTok{)}\OperatorTok{+}
\StringTok{  }\KeywordTok{scale_color_distiller}\NormalTok{(}\DataTypeTok{palette =} \StringTok{"Blues"}\NormalTok{, }\DataTypeTok{direction =} \DecValTok{1}\NormalTok{)}\OperatorTok{+}
\StringTok{  }\KeywordTok{scale_x_date}\NormalTok{( }
    \DataTypeTok{date_breaks =} \StringTok{"5 year"}\NormalTok{, }\DataTypeTok{date_labels =} \StringTok{"%y"}\NormalTok{) }
\end{Highlighting}
\end{Shaded}

\begin{verbatim}
## Warning: The labeller API has been updated. Labellers taking `variable`and
## `value` arguments are now deprecated. See labellers documentation.
\end{verbatim}

\includegraphics{A05_DataVisualization_files/figure-latex/unnamed-chunk-4-1.pdf}

\begin{Shaded}
\begin{Highlighting}[]
\NormalTok{?label_value}
\end{Highlighting}
\end{Shaded}

\begin{enumerate}
\def\labelenumi{\arabic{enumi}.}
\setcounter{enumi}{5}
\tightlist
\item
  {[}USGS gauge{]} Plot discharge by date. Create two plots, one with
  the points connected with geom\_line and one with the points connected
  with geom\_smooth (hint: do not use method = ``lm''). Place these
  graphs on the same plot (hint: ggarrange or something similar)
\end{enumerate}

\begin{Shaded}
\begin{Highlighting}[]
\CommentTok{#6 I chose the mean discharge instead of max discharge}
\KeywordTok{library}\NormalTok{(gridExtra)}
\KeywordTok{library}\NormalTok{(ggpubr)}
\end{Highlighting}
\end{Shaded}

\begin{verbatim}
## Loading required package: magrittr
\end{verbatim}

\begin{Shaded}
\begin{Highlighting}[]
\NormalTok{Plot1 <-}\StringTok{ }\KeywordTok{ggplot}\NormalTok{(USGS.data,}\KeywordTok{aes}\NormalTok{(}\DataTypeTok{x=}\NormalTok{datetime, }\DataTypeTok{y=}\NormalTok{X84936_00060_}\DecValTok{00003}\NormalTok{))}\OperatorTok{+}
\StringTok{  }\KeywordTok{geom_point}\NormalTok{(}\DataTypeTok{size=}\FloatTok{0.5}\NormalTok{)}\OperatorTok{+}
\StringTok{  }\KeywordTok{geom_line}\NormalTok{()}\OperatorTok{+}
\StringTok{  }\CommentTok{#xlim(2000,2020)+ When i add this line, I get error message "Error in as.Date.numeric(value) : 'origin' must be supplied""}
\StringTok{  }\KeywordTok{ylab}\NormalTok{(}\StringTok{"Discharge"}\NormalTok{)}\OperatorTok{+}
\StringTok{  }\KeywordTok{xlab}\NormalTok{(}\StringTok{"Year"}\NormalTok{)}

\NormalTok{Plot2 <-}\StringTok{ }\KeywordTok{ggplot}\NormalTok{(USGS.data,}\KeywordTok{aes}\NormalTok{(}\DataTypeTok{x=}\NormalTok{datetime, }\DataTypeTok{y=}\NormalTok{X84936_00060_}\DecValTok{00003}\NormalTok{))}\OperatorTok{+}
\StringTok{  }\KeywordTok{geom_point}\NormalTok{(}\DataTypeTok{size=}\FloatTok{0.5}\NormalTok{)}\OperatorTok{+}
\StringTok{  }\KeywordTok{geom_smooth}\NormalTok{(}\DataTypeTok{method=}\StringTok{"auto"}\NormalTok{)}\OperatorTok{+}\StringTok{  }
\StringTok{  }\CommentTok{#xlim(2000,2020)+}
\StringTok{  }\KeywordTok{ylab}\NormalTok{(}\StringTok{"Discharge"}\NormalTok{)}\OperatorTok{+}
\StringTok{  }\KeywordTok{xlab}\NormalTok{(}\StringTok{"Year"}\NormalTok{)}

\KeywordTok{ggarrange}\NormalTok{(Plot1, Plot2, }\DataTypeTok{nrow=}\DecValTok{1}\NormalTok{, }\DataTypeTok{ncol=}\DecValTok{2}\NormalTok{)}
\end{Highlighting}
\end{Shaded}

\begin{verbatim}
## Warning: Removed 28049 rows containing missing values (geom_point).
\end{verbatim}

\begin{verbatim}
## Warning: Removed 28033 rows containing missing values (geom_path).
\end{verbatim}

\begin{verbatim}
## `geom_smooth()` using method = 'gam' and formula 'y ~ s(x, bs = "cs")'
\end{verbatim}

\begin{verbatim}
## Warning: Removed 28049 rows containing non-finite values (stat_smooth).
\end{verbatim}

\begin{verbatim}
## Warning: Removed 28049 rows containing missing values (geom_point).
\end{verbatim}

\includegraphics{A05_DataVisualization_files/figure-latex/unnamed-chunk-5-1.pdf}
Question: How do these two types of lines affect your interpretation of
the data?

\begin{quote}
Answer: The first plot where points are connected by lines does not
really help in interpreting the data because I cannot see any patterns
from the lines. The second plot is slightly better as i can see the
general trend of where most data points lie.
\end{quote}

\begin{enumerate}
\def\labelenumi{\arabic{enumi}.}
\setcounter{enumi}{6}
\tightlist
\item
  {[}ECOTOX Neonicotinoids{]} Plot the concentration, divided by
  chemical name. Choose a geom that accurately portrays the distribution
  of data points.
\end{enumerate}

\begin{Shaded}
\begin{Highlighting}[]
\CommentTok{#7 }
\KeywordTok{ggplot}\NormalTok{(Ecotox,}\KeywordTok{aes}\NormalTok{(}\DataTypeTok{y=}\StringTok{`}\DataTypeTok{Conc..Mean..Std.}\StringTok{`}\NormalTok{, }\DataTypeTok{x=}\NormalTok{Chemical.Name, }\DataTypeTok{col=}\NormalTok{Chemical.Name))}\OperatorTok{+}
\StringTok{  }\KeywordTok{geom_violin}\NormalTok{() }\OperatorTok{+}
\StringTok{  }\KeywordTok{ylab}\NormalTok{(}\StringTok{"Concentration"}\NormalTok{)}\OperatorTok{+}
\StringTok{  }\KeywordTok{xlab}\NormalTok{(}\StringTok{"Chemical Types"}\NormalTok{)}\OperatorTok{+}
\StringTok{  }\KeywordTok{theme_bw}\NormalTok{(}\DataTypeTok{base_size =} \DecValTok{9}\NormalTok{) }\OperatorTok{+}
\StringTok{  }\KeywordTok{theme}\NormalTok{(}\DataTypeTok{legend.position=}\StringTok{"none"}\NormalTok{)}
\end{Highlighting}
\end{Shaded}

\includegraphics{A05_DataVisualization_files/figure-latex/unnamed-chunk-6-1.pdf}


\end{document}
